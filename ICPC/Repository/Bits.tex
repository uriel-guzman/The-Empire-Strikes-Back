\nsection{Bit tricks}

\subsection{Xor Basis}
\ncomment{Keeps the set of all xors among all possible subsets}
\addfile{../Codes/Math/Xor Basis.cpp}

\begin{tabular}{ |p{3cm}|p{5cm}|  }
  \hline  
  \rowcolor{Blue} 
  \multicolumn{2}{|c|}{Bits++} \\
  \hline
  \rowcolor{LightBlue2} 
  Operations on $int$ & Function \\
  \hline
  \texttt{x \& -x} & Least significant bit in $x$ \\
  \rowcolor{Gray} 
  \texttt{\_\_lg(x)} & Most significant bit in $x$ \\
  \texttt{c = x\&-x, r = x+c; (((r\^{}x) >> 2)/c) | r} & Next number after $x$ with same number of bits set \\
  \hline

  \rowcolor{LightBlue2} 
  \_\_builtin\_ & Function \\
  \hline 
  popcount(x) & Amount of 1's in $x$ \\
  \rowcolor{Gray} 
  clz(x) & 0's to the \textbf{left} of biggest bit \\
  ctz(x) & 0's to the \textbf{right} of smallest bit \\
  \hline
\end{tabular}

\subsection{Bitset} 
\vspace{-5pt}

\begin{tabular}{ |p{3cm}|p{5cm}|  }
  % \rowcolors{3}{gray}{white}
  \hline 
  \rowcolor{Blue} 
  \multicolumn{2}{|c|}{Bitset<Size>} \\
  \hline
  \rowcolor{LightBlue2} 
  Operation & Function \\
  \hline
  \_Find\_first() & Least significant bit \\
  \rowcolor{Gray} 
  \_Find\_next(idx) & First set bit after index $idx$ \\
  any(), none(), all() & Just what the expression says \\
  \rowcolor{Gray} 
  set(), reset(), flip() & Just what the expression says x2 \\
  to\_string('.', 'A') & Print 011010 like .AA.A. \\
  \hline
\end{tabular}